\section{Feasibility Study}
The feasibility study is done on the following aspects.
\begin{itemize}
    \item Operational Feasibility
    \item Technical Feasibility
    \item Economical Feasibility
    \item Schedule Feasibility
    \item Legal and Ethical Feasibility
\end{itemize}

\subsection{Operational Feasibility}
This section evaluates how well the proposed solution meets user requirements to address existing system issues. It is essential for the solution to meet these requirements to be operationally feasible. Currently, many activities related to community engagement and volunteerism are managed offline or through fragmented social media channels, leading to inefficiencies and missed opportunities.

Our platform aims to streamline these processes by centralizing and enhancing coordination among requesters, organizations, volunteers, and sponsors. This approach is designed to meet the needs of all stakeholders by providing a structured and accessible means of communication and collaboration.

The solution is a web application accessed via the Internet. To use the system, users must have:
\begin{itemize}
\itemsep0em 
    \item An internet connection.
    \item A mobile phone or a computer with a recent web browser installed.
    \item Basic IT and internet usage skills.
\end{itemize}

The platform's operational feasibility is supported by several factors:

\begin{itemize}
\itemsep0em 
    \item Individuals in need will readily seek assistance, making this application a crucial medium for addressing their needs. Therefore, requesters will be motivated to utilize it.
    \item Organizations will find value in this application as it provides easy access to a variety of volunteering opportunities. It also facilitates direct connection with sponsors to support their initiatives.
    \item Sponsors benefit from increased visibility and a platform for advertisement through the system. By spotlighting sponsors and organizations, the system enhances their public image.

\end{itemize}

Overall, the operational feasibility of the platform is bolstered by its ability to efficiently meet user requirements, ease of use, and its potential to enhance community engagement and support initiatives.


\subsection{Technical Feasibility}

This section assesses the practical implementation flexibility of our web-based platform using selected technical solutions. The core deliverable of our project is a web application constructed with HTML, CSS, and JavaScript for the front-end, and PHP for the back-end integrated with MySQL database. \\

The platform incorporates essential technologies:
\begin{itemize}
\itemsep0em 
    \item \textbf{SMS Gateway:} Enables the sending of notifications to users via text messages.
    \item \textbf{SMTP:} Utilized for sending transactional emails.
    \item \textbf{UI/UX Prototyping:} Figma will be used for designing and prototyping the user interface.
    \item \textbf{Diagramming and Modeling:} Draw.io and Lucidchart are employed for architectural planning and system modeling.
    \item \textbf{Version Control:} GitHub serves as our collaborative code repository and version control tool.
\end{itemize}

Our development approach emphasizes open-source and freely available technologies, ensuring cost-effectiveness and accessibility throughout the project. The team will acquire necessary technical expertise both prior to and during the development phase, supported by ample time within our timeline to build and expand our knowledge base.

The system architecture is designed to be cloud-ready, facilitating deployment on any cloud provider without dependence on proprietary or vendor-locked technologies. This ensures scalability and flexibility as the platform grows and evolves.\\

\subsection{Economical Feasibility}

The cost-effectiveness of the system will be evaluated in the economic feasibility study as follows: \\ \\
\textbf{Cost}
\begin{itemize}
\itemsep0em 
    \item Development Team Cost: The development costs are negligible as the software is being developed by university students as a volunteer project. Hence, there are no financial expenditures associated with the development team.
    \item Hosting Expenses: Given that the system is a web application, it necessitates a hosting fee. This fee covers server expenses to ensure continuous online accessibility of the system.
    \item Hardware and Software Expenses: No software costs will be incurred as the project utilizes free and open-source software. Additionally, hardware costs are avoided since development team members utilize their personal computers, laptops, and tablets. Existing networking equipment accessible to all team members will also be utilized.
    \item Domain Name Costs: The application is intended to be hosted on danaconnect.lk, with the domain name tracknbook.lk. The cost for the domain will be part of a standard full domain package available at LKR 5000 annually.
\end{itemize}
\newpage
\textbf{Revenue}

\begin{itemize}
    \item Commission: A small transaction fee or commission for successful matches facilitated through your platform.
    \item Partnerships: Collaborate with other platforms or organizations for joint initiatives or services that can be monetized.

%     \begin{table}[ht]
%     \centering


%     \begin{tabular}{|c|c|}
%         \hline
%         Expenses & Amount (LKR) \\
%         \hline
%         Hosting & 20,000.00  \\
%         Domain and SSL & 5,000.00 \\
%         SMS API & 25,000.00 \\
%         \hline
%         \textbf{Total} & \textbf{50,000.00} \\
%         \hline
%     \end{tabular}
%     \label{tab:expenses}
%     \caption{Expenses (yearly)}
% \end{table}
\end{itemize}
*Please note that the above shown figure is only a rough estimate. \\ 

Given the modest costs associated with hosting, domain registration, and upkeep, the dana System is confidently identified as economically viable. Developed through voluntary efforts of university students, and with minimal hardware and software expenses, the system is highly cost-effective. By aiming to deliver valuable services to train passengers without imposing substantial financial strain, dana represents a promising and financially sustainable solution.

\subsection{Schedule Feasibility}
The development of our project aims to be completed within April 2022, utilizing an iterative waterfall approach to ensure systematic progress and adherence to project milestones. For a comprehensive breakdown of our development timeline, please refer to the Project Schedule section. Our objective is to maintain schedule feasibility and deliver a robust, operational system within the designated timeframe.
\begin{itemize}[label={}]
\itemsep0em 
\item Working hours per week =  5 hours on weekdays + 5 hours on weekends = 10 hours
\item Number of group members = 4 members 
\item Number of weeks = 36 weeks
\item Total man hours = 10 * 4 * 36 hours = 1440 hours 
\end{itemize}


\subsection{Legal and Ethical Feasibility}
This section measures the legal and ethical issues we would be facing when building and implementing our platform and our countermeasures for that.
\begin{itemize}
    \item Data Protection: User data security is paramount, with stringent encryption and compliance with GDPR and local privacy laws.
    \item Intellectual Property: Users retain ownership of their content; proper attribution and permissions are enforced.
    \item Regulatory Compliance: Adherence to laws governing nonprofit operations, fundraising, and volunteer management.
    \item Financial Transactions: The system does not handle financial transactions. Any financial disputes or obligations are the responsibility of the organizations involved.
    \item Ethical Conduct: The platform promotes transparency, fairness, and accountability. Discrimination and harassment are strictly prohibited.
    \item Monitoring and Adaptation: Regular audits ensure ongoing compliance with legal requirements and best practices.
    \item User Responsibilities: Users are expected to adhere to community guidelines and ethical practices.
    % \begin{itemize}[label={}]
    % \item Article 14 of the Sri Lankan Constitution and the Personal Data Protection Act No. 9 of 2022 serve as the primary legal framework for personal data protection in Sri Lanka. Our system adheres to the stipulations outlined in these Acts, emphasizing the significance of ensuring user trust and maintaining information privacy within online transactions and information networks.
    % \end{itemize}
    
\end{itemize}